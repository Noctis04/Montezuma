\section*{ВВЕДЕНИЕ}
\addcontentsline{toc}{section}{ВВЕДЕНИЕ}


В мире современной информационной технологии игры стали неотъемлемой частью нашей жизни. В последние десятилетия создание игр стало одной из самых популярных и востребованных областей программирования. Искусство разработки игр предоставляет уникальную возможность соединить технические навыки с креативностью и фантазией. В рамках данной курсовой работы мы погрузимся в увлекательный мир разработки игр и рассмотрим процесс создания платформер-игры Montezuma’s Revenge с использованием Python и библиотеки SDL2.

Montezuma’s Revenge - игра, где главный персонаж перемещается по различным платформам, собирая предметы и преодолевая препятствия. Создание такой игры позволит нам познакомиться с различными аспектами программирования, начиная от разработки игровой механики и управления персонажем, и заканчивая обработкой графики.

В данной курсовой работе мы рассмотрим шаги, необходимые для создания собственной платформер-игры с использованием языка программирования Python и библиотеки SDL2 для работы с графикой. Мы изучим основные концепции и инструменты, необходимые для разработки игр, и погрузимся в мир 2D-геймдева, создавая интерактивное игровое приключение.


\emph{Цель настоящей работы} –  познакомиться с основами разработки игр и представить практический пример создания платформер-игры Montezuma’s Revenge на Python. Для достижения поставленной цели необходимо решить \emph{следующие задачи:}
\begin{itemize}
\item провести анализ предметной области;
\item разработать концептуальную модель игры;
\item спроектировать игровой движок с использованием Python и библиотеки SDL2;
\item реализовать на написанном движке игру.
\end{itemize}

\emph{Структура и объем работы.} Отчет состоит из введения, 4 разделов основной части, заключения, списка использованных источников, 2 приложений. Текст выпускной курсовой работы равен \formbytotal{page}{страниц}{е}{ам}{ам}.

\emph{Во введении} сформулирована цель работы, поставлены задачи разработки, описана структура проекта, приведено краткое содержание каждого из разделов.

\emph{В первом разделе} на стадии описания технической характеристики предметной области приводится сбор информации о игре Montezuma’s Revenge.

\emph{Во втором разделе} на стадии технического задания приводятся требования к разрабатываемому проекту.

\emph{В третьем разделе} на стадии технического проектирования представлены проектные решения для игры.

\emph{В четвертом разделе} приводится список классов и их методов, использованных при разработке игры, производится тестирование разработанного проекта.

В заключении излагаются основные результаты работы, полученные в ходе разработки.

В приложении А представлен графический материал.
В приложении Б представлены фрагменты исходного кода. 
